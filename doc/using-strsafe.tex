\section{Using the \strsafe{} functions}
The functions in \strsafe{} are meant as replacement for
the string copying functions available in the standard C libraries.
In contrast to the standard functions,
the functions always require the maximum capacity of any buffer
that the funciton will write data into
and will guarantee both
that the string will be null terminated before the end of the buffer
and that no data will be written outside of the buffer.

To simplify use of the library,
all functions share a common system of return codes.
All functions return a status code signifying the result of the call.
These status codes can be used to check if the call was successful
or check the specific error condition if it was not.

\subsection{Choice of functions}
There are several choices of functions
that perform similar operations in \strsafe{}.
The distinctions are in the character representation used
and whether to measure the sizes of buffers
in number of characters or bytes.

\subsubsection{String representation}
\strsafe{} provides functions that work with standard \type{char} strings.
It also provides the same functionality for \type{wchar\_t} strings,
often used with \emph{Unicode} encodings.
Finally, \strsafe{} also provides a \type{TCHAR} type that can be either a
\type{wchar\_t} or a \type{char} depending on
whether the macro \define{UNICODE} is set or not.

To use \strsafe{} with \type{char} strings,
use the functions ending with \fn{A}.
To use \strsafe{} with \type{wchar\_t} strings,
use the functions ending with \fn{W}.
To use \strsafe{} with \type{TCHAR} strings,
use the functions that have neither \fn{A} nor \fn{W} at the end.

\subsubsection{Specifying buffer capacities}
The capacity of a string buffer can be specified either in the number of
characters or bytes that the buffer can hold.
By using the functions beginning with \fn{StringCch},
all buffer sizes will be interpreted as a number of characters.
By using the functions beginning with \fn{StringCb},
all buffer sizes will instead be interpreted as a number of bytes.
If this size is not divisible by the size of a single character,
the size will be rounded down to the nearest multiple of the character size.

By definition, a \type{char} has size one.
This means that there is no difference between the two types of functions
when using \type{char} strings.

\clearpage
\subsection{Return codes}
All functions return a \define{HRESULT} value.
The possible return values are:

\begin{table}[htb]
\begin{tabularx}{\textwidth}{lX}
\toprule
Return value & Description \\
\midrule
\define{S\_OK} &
Function ran successfully.
No error occured during the function execution. \\
\define{STRSAFE\_E\_END\_OF\_FILE} &
The end of an input file or stream was reached during execution. \\
\define{STRSAFE\_E\_INVALID\_PARAMETER} &
One or more of the input parameters was invalid. \\
\define{STRSAFE\_E\_INSUFFICIENT\_BUFFER} &
An output buffer used by the function was too small
to contain the data that was supposed to be written
into it during the function call.
The data in the output buffer has been truncated and null terminated. \\
\bottomrule
\end{tabularx}
\caption{\strsafe{} function return values.}
\end{table}

The only return value to indicate success is \define{S\_OK}.
To check the return value the macros
\define{SUCCEEDED} and \define{FAILED} are provided.
Depending on the specific use case,
a result of \define{STRSAFE\_E\_INSUFFICIENT\_BUFFER}
might not be regarded as a failure by the user.
If that is the case, it is up to the user to check the returned value.
